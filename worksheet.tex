\documentclass{dcbl/challenge}

\setdoctitle{Non-Trivial Types}
\setdocauthor{Stephan Bökelmann}
\setdocemail{sboekelmann@ep1.rub.de}
\setdocinstitute{AG Physik der Hadronen und Kerne}
\usepackage{listings}

\lstset{language=C, basicstyle=\ttfamily}
\usepackage{hyperref}

\begin{document}
Our task as programmers is it to encode status into memory and processes into transformations.
We already know how to write functions that can transform our data in memory. But what if a representation of our status is more complicated than just a number?

A window, for example, needs a width and a height to be stored alongside each other.
Data types that define the structure of that are what we call \textit{non-trivial types}.
These types are not limited to numbers, and we will explore several possible non-trivial types in this exercise.
\begin{enumerate}
    \item structs
    \item pointers
    \item arrays and
    \item function-pointers (this will be on a different worksheet)
\end{enumerate}

\section*{Exercises}
\begin{aufgabe}
    Let's start with a struct. 
    A struct is a collection of fields, addressable by a common identifier.
    
    Say your program want's to handle the data of two different persons. 
    How would you save the data in memory?
    One way could be the following:
    \begin{lstlisting}
        int age_person_1 = 20;
        int age_person_2 = 30;
        int weight_person_1 = 70;
        int weight_person_2 = 80;
        int height_person_1 = 180;
        int height_person_2 = 170;
    \end{lstlisting}
    But this is not very readable and quite error-prone to missing something. Furthermore, it is not very easy to pass an entire person as an argument to a function.
    
    A better way to implement this are structs. 
    First, we need to define a struct so that the compiler knows what information that struct holds:
    \begin{lstlisting}
        struct Person_t {
            int age;
            int weight;
            int height;
        };
    \end{lstlisting}
    We use the \texttt{\_t} suffix to indicate that this is a user-defined type.
    Let's refactor our code from before:
    \begin{lstlisting}
        struct Person_t person_1 = {20, 70, 180};
        struct Person_t person_2 = {30, 80, 170};
    \end{lstlisting}
    
    But there are even more things we could do to make the code more readable. 
    \begin{lstlisting}
        typedef struct Person_t P_t;
        P_t person_1 = {20, 70, 180};
        P_t person_2 = {30, 80, 170};
    \end{lstlisting}
    This typedef syntax makes it easier to read the code, since we can neglect the \texttt{struct}-keyword in our code.
    A typedef is also called an \textit{alias}. Here \texttt{P\_t} was defined as an alias for \texttt{struct Person\_t}.
    
    But there is another improvement that we can do to make our code more readable:
    \begin{lstlisting}
        typedef struct{
            int age;
            int weight;
            int height;
        } P_t;
        P_t person_1 = {
            .age = 20,
            .weight = 70,
            .height = 180
        };
        P_t person_2 = {
            .age = 30,
            .weight = 80,
            .height = 170
        };
    \end{lstlisting}
    Now this looks really readable, and every programmer can understand what this code does. 
    We can access the fields using dot-notation.
    \begin{lstlisting}
        printf("%d", person_1.age); 
    \end{lstlisting}
    We can also pass a struct as an argument to a function.
    \begin{lstlisting}
        void print_person(P_t person) {
            printf("%d %d %d", person.age, person.weight, person.height);
        }
    \end{lstlisting}
    We can also return a struct from a function.
    \begin{lstlisting}
        P_t get_person() {
            return person_1;
        }
    \end{lstlisting}
    This makes handling data a lot easier, since we can keep data together in one place and pass it around by only referencing the struct, rather than each field separately.
    
    Implement a struct, that encodes the information of a car!    
\end{aufgabe}

\begin{aufgabe}
    The next data type that we want to talk about are pointers.
    A pointer is essentially a type that stores the memory address of another piece of data. 
    (What size does a pointer have? It depends on your architecture: For 32-Bit architectures, each address has 32 bit - hence the pointer needs that much space. For 64-Bit architectures, this changes to 64 bit)
    
    This might seem abstract at first, but pointers are a powerful feature of the C language, allowing for efficient and dynamic memory management. 
    Consider you want to modify the value of a variable inside a function:
\begin{lstlisting}
    void add_ten(int n) {
        n += 10;
    }
\end{lstlisting}
If you pass a variable to this function, the original variable will not be modified because C passes arguments by value. 
Here's where pointers come into play. 
By passing a pointer to the variable, you can directly modify the variable's value:
\begin{lstlisting}
    void add_ten(int *n) {
        *n += 10;
    }
\end{lstlisting}
In this version, \texttt{n} is a pointer to an integer, and \texttt{*n} accesses the value of the variable it points to, allowing us to modify the original variable.
It's important to recognize that a pointer, on its own, does not constitute a complete type. Its definition is only complete when it is associated with the type of data it references.
This is why we call it a \textit{pointer to int} or \textit{pointer to float}, or whatever it is pointing to.

To use this function, you would pass the address of a variable to it, like so:
\begin{lstlisting}
    int main() {
        int age = 20;
        add_ten(&age);
        printf("%d", age); // Outputs: 30
    }
\end{lstlisting}
The \texttt{\&} operator gets the address of \texttt{age}, and the function \texttt{add\_ten} modifies the value at that address. 
The value of the \texttt{age} variable in \texttt{main} is now 30.

Pointers are not limited to primitive data types. They can also point to structures, enabling the manipulation of complex data structures:
\begin{lstlisting}
    typedef struct {
        int x;
        int y;
    } Point_t;

    void move_point(Point *p, int dx, int dy) {
        p -> x += dx;
        p -> y += dy;
    }
\end{lstlisting}
In this example, \texttt{p} is a pointer to a \texttt{Point} structure, and \texttt{p → x} is the syntax used to access the \texttt{x} field of the structure \texttt{p} points to. This function modifies the position of a point.

Pointers are a fundamental aspect of C, enabling direct memory access and manipulation, which is critical for system programming, efficient data structures, and interfacing with hardware.

Explore pointers further by creating a simple program that uses pointers to swap the values stored behind two identifiers.
Make your function take two pointers as arguments.
Take a look at the following example before proceeding: \href{https://godbolt.org/#z:OYLghAFBqd5QCxAYwPYBMCmBRdBLAF1QCcAaPECAMzwBtMA7AQwFtMQByARg9KtQYEAysib0QXACx8BBAKoBnTAAUAHpwAMvAFYTStJg1DIApACYAQuYukl9ZATwDKjdAGFUtAK4sGIAMxmpK4AMngMmAByPgBGmMQgZkEADqgKhE4MHt6%2BASlpGQJhEdEscQlJtpj2jgJCBEzEBNk%2BfoFVNZn1jQTFUbHxiUEKDU0tue0jPX2l5UMAlLaoXsTI7Bzm/uHI3lgm/m4j%2BKgAdAj72CYaAIKb27uY%2B4cE6DGonmcXV7c33%2BEEAGoWExwhB5iYAOxWG4A2EA/5wgEAN0aXAB%2BwAIgCuPtoddEQoECRAcjGmZ0f4sVwzLjvoiEXCUcR/BSqWZ/LSYXDaAJgIzGpJWdj2ZJOfjubyATyjKTiABWIXU/ySOVixEAfQs71osoAbIrcQCAPRGgEAdSYCDIAPQTD1AIYeGQCEBGnhIzpcKoPKYgKZEMV7JOKt1arh6GWMXosoAHIHlSc5bqITGxZ7YcliP8qBBzGYzfE/QJ0XK3Bp/By5RZJBpcVwWLK0YjYSB0WY5egS2WKyYqxoWL23Aw86RG%2BD/HjEZns7mktd0MRMAolMji4Py5Xq7WJ/XG82Aa3zHLkl2N72LP3B8PKm3dUyuOPJ3Dp4Ic3mC01VwxTz2q5JRTuDZMuS%2B6Hu2nbrr%2BF4DqW15BLKZiPumAIvgQb5zguS4rkia6lmef4ARYu7AaBbbHj%2Bm6XrBI63sBSFchmWavrO%2BaFl%2BFHniqdZAY0LKkUeEF4VBVFDjRTL%2BPR4qMTOebzouy6YOxkGblxgGynxiJgeRynniJcGjuYd68ZJU5MWhLEfkW346X%2BOJqUygr8e2tCCd2lEwaJN4OSZz5mehZhyVhik4dZQkqXZRE8cQjmaWRJ42dBV40YZ3lhtJzHvmxIUcX%2B26RbKCpOXKtAuTliXUV5jRyj56XmbJmEKUpYWcXlxFVUV8XNX2Hn6bRVU1ShfkWVluFuZxhFtcQ%2BpFa5%2BHlZ58FMrqA2of5gWNdlCX/tx9rNlpnVjd1SU3iljTLWlg0yUkllNYd1aqvZjQBkVVBlXpYlPStQ31fJ2GjXNqn5f6HVvT1yVmEZxAQl9V2sZ%2Bm1ddWEWTXGRW0K9CXvZVxAxjDGUYb9wX/VBkjI1FqOxUeB1zVj8GnTjknIYuBArN%2BeXfJCGLfBwiy0Jwcq8H4HBaKQqCcG41jWACCjLKsimbDwpAEJoPOLAA1iAcoAJwnGYKa6lwWta7qMa6v4SYqvonCSILKui5wvAKCAGhKyrixwLASCYKomDIF4RBkBQECNMACjKIY1RCESADuQuK2gLDJHQvqZOHES0FHqCx3bCdJ/QCTAFwGhBLndDxJErDrLwpf5wA8v7mfZ8LvDe771zEKHDvBD7yD1PgQu8PwggiGI7BSDIgiKCo6jN6QuhcPohjGJLlj6HgMRO5AiyoMktQME7HC8KgSLxFmWCb2CpDEF4gh4GwAAq2oX4sMsrGsehHOEaeRzHce8NHxAmDJE4DwXm/NbazzFhwbAPd/YkABKoU2ABaXUgpgDIGQNiDQusAQQAlpYawo5cCEHgQreYvBlbN3mOrEANYTgQjMFwfwMYNDliYag82corYcBtqQIWIsoGO2dq7KhYCOBmAgQIrulCtDUNICfYg6RnCSCAA%3D%3D%3D}{Pointer Example on Godbolt}

\end{aufgabe}

\begin{aufgabe}
    Moving forward, let's explore the relationship between pointers and arrays in C. Both are closely related, as arrays in C are essentially a sequence of contiguous memory locations, and the name of an array acts as a pointer to its first element.
    Consider you have an array of integers:
\begin{lstlisting}
    int numbers[3] = {10, 20, 30};
\end{lstlisting}
The name \emph{numbers} points to the first element of the array, meaning \emph{numbers} is equivalent to \emph{\&numbers[0]}. You can verify this by comparing their addresses:
\begin{lstlisting}
    printf("Address of numbers: %p\n", (void *)numbers);
    printf("Address of &numbers[0]: %p\n", (void *)&numbers[0]);
\end{lstlisting}
You'll notice that both addresses are the same, so they refer to the same memory on your computer and will always reference the same value. This illustrates how array names can be thought of as pointers.

Accessing array elements using pointers is straightforward. You can access the first element of the array `numbers` by dereferencing the pointer:
\begin{lstlisting}
    printf("First element: %d\n", *numbers);
\end{lstlisting}
To access the second element, you can take advantage of pointer arithmetic. Instead of using a loop, let's increment the pointer by 1 (which actually adds the size of the type it points to, in this case, `int`):
\begin{lstlisting}
    printf("Second element: %d\n", *(numbers + 1));
\end{lstlisting}
Similarly, for the third element:
\begin{lstlisting}
    printf("Third element: %d\n", *(numbers + 2));
\end{lstlisting}
This method of accessing array elements highlights the duality between pointers and arrays in C. You directly use the array name as a pointer and apply arithmetic operations to navigate through the array.

While pointers and arrays appear similar, it's crucial to understand their differences. An array name is a constant pointer to the first element and cannot be reassigned, whereas a pointer is a variable that can point to different memory locations during its lifetime. This distinction underlines the flexibility of pointers compared to array names.

Experiment further by creating an array of your \texttt{Car\_t}. 
Create an array, holding three cars. 
Write a function, that takes a pointer to a particular member-value as an argument.
Find out, how to pass the address of the member-value to the function.

\end{aufgabe}

\section*{Anmerkungen}

\end{document}
